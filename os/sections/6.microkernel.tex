\chapter{\textit{Microkernel}}
A palavra \textit{kernel} é tradicionalmente utilizada para denotar a parte do SO que é essencial e comum a todos os outros \textit{softwares}. A idéia principal da abordagem do \textit{microkernel} é reduzir o tamanho \textit{kernel}, implementando o maior número possível de funções e sub-sistemas fora dele. Esta idéia não é nova, advinda desde 1970. A maioria dos \textit{microkernels} comerciais utilizam a abordagem cliente/servidor. \underline{Exemplos:} Mach, Amoeba e Chorus.

A interface oferecida garante uma estrutura de sistema mais modular, de forma que o mal-funcionamento de servidores de funções pode ser isolado, como em qualquer outro programa de usuário, protegendo o \textit{kernel} de ser afetado. Além disso, isso permite que diferentes estratégias implementadas por diferentes servidores podem coexistir, garantindo \textbf{maior flexibilidade}.

Entretanto, esta abordagem apresenta a desvantagem inata de possuir \textbf{baixo desempenho}. Tal característica provêm dê:
\begin{itemize}
  \item Aumento do número de tarefas que rodam no modo usuário;
  \item Aumento no número da troca de contextos;
  \item Mecanismo de IPC utilizado entre processos: \textit{send} e \textit{receive}.
\end{itemize}

Com o objetivo de mitigar este problema, a flexibilidade da arquitetura foi sacrificada, para que diversos serviços fossem re-integrados ao \textit{kernel}, evitando as trocas de contexto. Dessa forma, os \textbf{mecanismos que permaneceram no \textit{microkernel} foram}:
\begin{itemize}
  \item Gerência de processos e \textit{threads};
  \item Comunicação de processos;
  \item Proteção e gerência de memória dependente do \textit{hardware};
  \begin{itemize}
    \item Gerênia de memória independente do \textit{hardware} e operações de \textit{page-in} e \textit{page-out} ficam de fora.
  \end{itemize}
\end{itemize}


\subsection{Tratamento de Entrada e Saída}
O espaço de endereçamento é a abstração óbvia para a incorporação de portas de dispositivos. Como o espaço de endereçamento é controlado pelo gerente de memória, o controle de permissão de E/S e os \textit{drivers} de dispositivos \textbf{podem ser implementados fora do \textit{kernel}}.

O \textbf{tratamento de interrupção} neste caso é diferente: o \textit{kernel} recebe a interrupção, transforma-a em mensagem e a envia ao \textit{driver} correspondente, que roda em modo usuário. Desta forma, \textbf{\textit{kernel} não se envolve no tratamento específico de uma interrupção}.

Por sua vez, o \textit{driver} é um processo que acessa diretamente portas de \textit{hardware} em seu espaço de endereçamento e recebe as interrupções via mecanismos de IPC. Caso necessite manipular a memória, pode fazê-lo por meio de gerentes de memória. \underline{Exemplo:} \textit{driver} de vídeo.

\subsection{Identificadores Únicos}
O \textit{microkernel} provê identificadores únicos (\textit{uid}) para recursos como \textit{threads}, tarefas, canais de comunicação, etc.. Logo, se um recurso $S_1$ quer enviar uma mensagem ao recurso $S_2$, ele deve especificar $S_2$ como destino. Para tal, o \textit{microkernel} deve conheces estes \textit{uid}.

\subsection{Gerentes de Memória}
Um servidor que gerencia o espaço de endereçamento $E0$ será um gerente de memória tradicional, porém fora do \textit{microkernel}. A memória física a ser gerenciada será mapeada para o espaço de endereçamento do servidor.

\subsection{Paginadores}
Os paginadores utilizam as primitivas \textit{grant}, \textit{map} e \textit{flush}, oferecidas pelo \textit{microkernel}. As interfaces entre paginador e cliente, paginador e servidor e paginador e \textit{drivers de dispositivos} são baseadas em IPC e definidas a nível de usuário.









\section{Exemplo: Mach}

\subsection{Modelo de Memória}
No Mach o \textbf{modelo de memória} é flexível e independente do \textit{hardware}. Ele divide a gerência de memória em três partes: \textit{pmap}, modo \textit{kernel} independente de arquitetura e paginador externo.

\subsubsection{\textit{Pmap}}
Este modelo é o modo \textit{kernel} dependente da arquitetura, onde o sistema Mach admite que o processador possui uma unidade de gerência de memória (MMU), onde o \textit{pmap} é o módulo do que controla e acessa a MMU. Desta forma, o \textit{pmap} é dependente do \textit{hardware}.

\subsubsection{\textit{Kernel} Independente de Máquina}
Este módulo trata da parte do processamento do \textit{page fault}, que é independente de máquina. Ele realiza o mapeamento de endereços nas tabelas do sistema e a substituição de páginas.

\subsubsection{Paginador Externo}
Trata da lógica de funcionamento da memória virtual. Portanto, faz:
\begin{itemize}
  \item Gerência da tabela de páginas: cuidando de páginas alocadas e livres, estados das páginas, etc.;

  \item Gerência da atividade de \textit{page-in} e \textit{page-out}: decide quando uma página deve ser carregada em memória (demanda, \textit{prefetching}) e libera o espaço ocupado em memória pela página quando a mesma é escrita em disco.
\end{itemize}

Como o paginador externo roda em modo usuário, várias outras funções podem ser incorporadas a ele.


\subsection{Memória Virtual}
O espaço de endereçamento de um processo é paginado. Este espaço é dividido em unidades esparsas de tamanho variável, chamadas \textbf{regiões}. Cada região que estive mapeada em memória de um processo é denominada \textbf{objeto de memória}. Eles podem ser arquivos, \textit{pipes}, etc..

Os objetos de memória podem ser gerenciados tanto por paginadores externos como por paginadores \textit{default} do Mach. O objeto de memória é referenciado por uma porta e mensagens são enviadas para solicitar operações sobre o objeto.

Portanto, são definidas primitivas para a memória virtual:
\begin{itemize}
  \item \textbf{\textit{vm_allocate}:} torna a região de memória utilizável, alocando espaço em uma áreea de memória já mapeada;

  \item \textbf{\textit{vm_deallocate}:} retira uma região do mapa de memória utilizável;

  \item \textbf{\textit{vm_map}:} mapeia uma região de memória no espaço de endereçamento do processo. Logo, ela:
  \begin{itemize}
    \item Retorna um ponteiro para o início da região mapeada;

    \item Neste momento, a região é mapeada, porém não é carregada em memória. Assim, o primeiro acesso a esta região ocasionará em um \textit{page fault}.
  \end{itemize}

  \item \textbf{\textit{vm_copy}:} copia um objeto da memória de uma região par a outra. Como otimização, ele implementa o \textit{copy on write}, onde % TODO: ver o que é isso e por figura

  \item \textbf{\textit{vm_inherit}:} determina que regiões serão herdadas na criação de um processo filho;

  \item \textbf{\textit{vm_read}:} lê de uma região mapeada em outro espaço de endereçamento;

  \item \textbf{\textit{vm_write}:} escreve uma região mapeada em outro espaço de endereçamento;

\end{itemize}

\subsubsection{Paginadores Externo}
Objetos de memória são necessariamente paginados, podendo ser gerenciados por aplicações de usuário chamadas gerentes de memória ou paginadores externos. Os gerentes de memória são os responsáveis pelas atividades de \textit{page-in} e \textit{page-out}.

% TODO: por figura

\subsubsection{Protocolo entre Gerente e Núcleo}
O gerente de memória interage com o núcleo do \textit{Mach} através de um protocolo assíncrono pré-definido:

\begin{enumerate}
  \item Uma tarefa faz o mapeamento (\textit{vm_map}) de um objeto de memória sobre o seu espaço de endereçamento. Nesse momento, a tarefa nomeia o gerente de memória como sendo responsável pela gestão do objeto;

  \item O núcleo envia uma mensagem ao gerente de memória (\textit{memory_object_init}), dizendo que o objeto que ele gerencia foi mapeado. O gerente faz algumas inicializações e responde ao núcleo com \textit{memory_object_ready};

  \item A tarefa é referencia o objeto de memória. As referências são feitas por operações básicas de acesso à memória (\textit{LOAD} e \textit{STORE}). Um \textit{page fault} é gerado. O núcleo é ativado e contacta o gerente de memória responsável pelo objeto de memória e pede a página por \textit{memory_object_data_request};

  \item Quando a página estiver disponível, o gerente de memória responde ao \textit{kernel} com \textit{memory_object_data_supply};

  \item A substituição de páginas é decidida pelo núcleo \textit{Mach}. Nesse caso, a página é enviada ao gerente de memóriam com um \textit{memory_object_data_return}, que é o responsável pelo seu armazenamento;

  \item Se a aplicação deseja referenciar a página de um modo proibido pelos direitos de acesso atuais, um \textit{page fault} de proteção é gerado. O gerente pode também decidir restringir os direitos de acesso à uma página por conta própria.
\end{enumerate}

% TODO: botar figura

É importante notar que o usuário pode não querer gerenciar um objeto de memória e deseja que o sistema o faça, uma vez que sendo uma aplicação de usuário, pode conter \textit{bugs}. As páginas que compõe o próprio sistema \textit{Mach} devem ser geradas por alguém. Dado isso, o sistema dispõe de um \textbf{gerente de memória \textit{default}}.

O modelo de memória do \textit{Mach} apresenta grande flexibilidade, porém seu desempenho é baixo. Atualmente, o Unix incorpora parte destas idéias através das primitivas \textit{mmap} e \textit{mprotect}.









\section{Outros Detalhes}

\subsection{Comunicação Remota}
Os \textit{microkernels} foram propostos inicialmente para ambientes distribuídos. O IPC remoto é implementado por servidores que traduzem mensagens locais para protocolos de comunicação externos e vice-versa.





\section{Modelos Alternativos}

\subsection{Ênfase em Portabilidade}
\textbf{\textit{Microkernels} antigos} eram construídos de maneira independente de \textit{hardware}, em cima de uma pequena camada dependente de \textit{hardware}. Esta última tem as seguintes implicações:
\begin{itemize}
  \item Características específicas do \textit{hardware} não podem ser aproveitadas;

  \item A camada de abstração de \textit{hardware} introduz um custo no desempenho.
\end{itemize}

Mesmo com todo este cuidado, em alguns momento, o \textit{microkernel} deve estar consciente das características do \textit{hardware}, que envolvem implementação do espaço de endereçamento do usuário. \underline{Exemplo:} registradores de segmento no Pentium ou a troca de contexto tradicional no i486.

Justamente por esta camada de abstração do \textit{hardware}, o desempenho nestas abordagens acabou por ficar lento.

Arquiteturas diferentes requerem técnicas de otimização específicas que muitas vezes afetam a estrutura global do \textit{microkernel}. Portanto, os \textit{microkernels} mais recentes não são portáveis, sendo a base dependente da arquitetura sobre a qual serão construídos sistemas operacionais portáveis.




\subsection{Domínios de Proteção}
É uma outra estruturação possível além do cliente/servidor, tendo como exemplo as implementações SPACE, Kea, Spring e Pebble. Elas são baseadas em duas abstrações: domínios de proteção e \textit{threads}. Um programa consiste de: um conjunto de domínios de proteção e uma única \textit{thread}.

O \textbf{domínio de proteção} é composto por um espaço de endereçamento e informações de proteção. Os módulos do sistema operacional são implementados como domínios de proteção. Cada domínio registra seus métodos e o mecanismo de comunicação inter-domínio (IDC) é utilizado para se obter acesso a um método protegido de outro domínio. Existe uma pilha de domínios de proteção que contém as IDCs em andamento.

Como \textbf{vantages} em relação a abordagem cliente/servidor, esta implementação reduz o \textit{overhead} da troca de mensagens e o da troca de contexto entre \textit{threads}. Como \textbf{desvantagem}, ele introduz o \textit{overhead} da pilha de domínio de proteção.






\section{Três Níveis de Proteção}
Outra estruturação possível, baseada nos níveis de proteção da arquitetura Intel x86, que tem 4 níveis de proteção. É implementada pelo \textit{Paladium}.

Em um dado momento, um processo está em um nível de proteção e só pode acessar dados no mesmo nível ou em níveis inferiores, dado que o código só pode ser acessado no mesmo nível de proteção.

Logo, são implementados os \textbf{call gates}. Cada nível de proteção define quais dos seus métodos são acessíveis através de um \textit{call gate}, onde o nível pode ter um número qualquer \textit{gates}. O \textit{call gate} por sua vez será utilizado para movimentação entre os níveis de proteção.

O espaço de endereçamento de um processo é divido em três níveis: \textit{kernel}, intermediário e usuário.

Cada módulo do SO independe do \textit{hardware} e é implementado como um módulo no nível intermediário. Um \textit{call gate} é definido, onde estão incluídos os métodos do módulo (interface). Assim, a comunicação é feita por \textit{call gates}.

Como \textbf{vantagem} em relação ao modelo cliente/servidor, esta implementação reduz o \textit{overhead} da troca de mensagens, troca de contexto entre \textit{threads} e da troca de espaços de endereçamento. Como \textbf{desvantagens}, vemos que esse modelo é dependente do \textit{hardware} dados os níveis de proteção e os módulos do SO não estão protegidos entre si.
