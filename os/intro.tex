\chapter*{Sobre este Documento}

Este resumo tem o objetivo de auxiliar o leitor na prova da matéria de Fundamentos de Sistemas Computacionais, da Profª Alba Melo, da Universidade de Brasília. Este documento foi gerado em \today{} às \currenttime. Para contribuições ou acompanhamento do material, acesse o repositório oficial: \url{https://github.com/rodopoulos/fsc-notes}.

Os \textit{slides} disponibilizados pela professora dispõe o conteúdo de uma forma objetiva, porém um pouco esparsa e, por vezes, vaga. Dessa forma, conceitos não são bem destrinchados, o que requer que o aluno tenha gravado totalmente a aula ou que busque melhores explicações nos livros citados na bibliografia do curso, o que é oneroso.

Dessa forma, este resumo tenta unir o melhor dos dois mundos: manter a objetividade dos \textit{slides}, aliado a uma razoável especificação dos livros da matéria: Tanenbaum e Silberchatz. A estratégia é manter o texto base dos slides, não alterando a conceituação inicial que é apresentada. Porém, ao ser constatado que faltam detalhes, existem ambiguidades ou poderiam ser melhorados, os conceitos são estendidos e melhor ilustrados.

É interessante utilizar o apêndice contendo exercícios como complemento. Os exercícios geralmente ajudam a conceituar alguns pontos da matéria por outra perspectiva, mesclando algumas partes do conteúdo.

Caso você encontre algum erro, seja conceitual, gramático ou ortográfico, envie o problema no repositório do projeto, ajudando a contribuir para melhorar este documento.

Boa sorte!
